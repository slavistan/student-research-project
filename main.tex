\documentclass{article}

% Packages
\usepackage[utf8]{inputenc} % Required for German umlauts

\author{Stanislaw Hüll}
\title{Student Research Project}
\date{2018-11-11}
\begin{document}

\maketitle

\section{Introduction}

  \subsection{Compressed Sparse Row Format}
    Consists of values array (A), column indices (JA), row pointers (IA)
    Makes no assumptions about the matrix in contrast to other formats (DIAGONAL)
      -> General purpose sparse matrix format
    Good properties for mat-vec-mult (locality)
    Compressed Sparse Column

  \subsection{Structured Grid Matrices}
    Arise as finite difference adjacency matrices from discretization of elliptic type PDE (SPARSKIT 7.1)
      and structured mesh finite element matrices.
    Are (locally) structured.

\section{Three-fold compressed sparse row}
  Improve upon CSR format utilizing domain knowledge about structured grid matrices.

  \subsection{Data layout and storage scheme}
    Starts out at a CSR representation. This section explains the general storage scheme. Its usage and motivation will be clarified in the next section.
    Applies compression idea of CSR to the non-zeros' value array and column index array separately (... motivation?).
    V stores the numerical value of each non-zero entry, VS stores an index-pointer into V to each initial element of a row's section within V.

      Example: Compressed and uncompressed values (Matrix <-> Varrays)

    Column index information is split into three arrays.
    Peg-index: Column index of each row's first non-zero. (JPeg)
    jrel: Column indices relative to their row's peg-index (:= 'index pattern') (JRel)
    jrelstart: Index-pointer into JRel to each initial element of a row's section within JRel.

      Example: Compressed and uncompressed column indices (Matrix <-> Varrays)

    High-level view of the data organization:
    (A) Data representing the matrix's non-zero elements
      A real array V & vstart;

    (B) Data representing non-zero elements' column indices
      jrel & jrelstart & peg-index

    (C) Row pointers (as in CSR) row-pointers

  \subsection{Compression mechanism}

  \subsection{Algorithms}

    \subsubsection{Matvecmult}
      improves locality (--> Performance)

    \subsubsection{Element access A[i;j]}
      binary lookup

\section{Performance Benchmarks}
  Measure performance in terms of (1) data compression ratio and (2) arithmetic performance (matvecmult)

  \subsection{Generation of structured grid adjacency matrices as test matrices}

  \subsection{Data compression}

  \subsection{Arithmetic performance (matvecmult)}

\section{Summary}
  Advantages: General purpose, good performance, parallel scalability
  Disadv: Static structure (cannot add/remove elements)

\end{document}
