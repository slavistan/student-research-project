\documentclass{article}

% Packages
\usepackage[utf8]{inputenc} % Required for German umlauts

\usepackage[backend=biber]{biblatex}
\addbibresource{bib/bibliography.bib}

%
% Tikz Setup
%
\usepackage[customcolors]{hf-tikz}

\tikzset{style green/.style={
    set fill color=green!50!lime!60,
    set border color=white,
  },
  style cyan/.style={
    set fill color=cyan!90!blue!60,
    set border color=white,
  },
  style orange/.style={
    set fill color=orange!80!red!60,
    set border color=white,
  },
  hor/.style={
    above left offset={-0.15,0.31},
    below right offset={0.15,-0.125},
    #1
  },
  ver/.style={
    above left offset={-0.1,0.3},
    below right offset={0.15,-0.15},
    #1
  }
}

\author{Stanislaw Hüll}
\title{Student Research Project}
\date{2018-11-11}
\begin{document}

\maketitle

\section{Introduction}

  \subsection{Compressed Sparse Row Format}
    It is a widely used storage format for sparse matrices which does not make assumptions about the matrix's shape
    or its distribution of non-zero elements in contrast to other popular sparse matrix storage formats such as the
    the diagonal storage format or ELLPACK. It optimizes the storage requirements of general sparse matrices with
    respect to the naive coordinate format (COO) in that each non-zero's row index is no longer explicitly stored.

    The CSR format consists of three dense arrays:

    The values array stores the numerical value for each non-zero entry in the matrix while their column index is stored
    in the associated column-index array. A third array, the row-pointer array, encodes the beginning of each row's section within the
    values and column-index arrays, i.e. it stores the offset of each row's first non-zero element into the two previous
    arrays. Thus the CSR format optimizes the storage requirements of general sparse matrices with respect to the naive
    coordinate format (COO) in that each-nonzero's row index is no longer explicitly stored, shrinking the size of the
    third array from one entry per non-zero element to a single entry per row.

    % CSR format example

\begin{equation}\label{fig:csr_example}
  \left(\begin{array}{cccc}
    \tikzmarkin[hor=style green]{row0}  3 & 0 & 4 & 0 \tikzmarkend{row0} \\
    \tikzmarkin[hor=style cyan]{row1}   0 & 0 & 0 & 0 \tikzmarkend{row1} \\
    \tikzmarkin[hor=style orange]{row2} 0 & 0 & 7 & 8 \tikzmarkend{row2} \\
  \end{array}\right)
\end{equation}


    By convention, the non-zeros are stored row-wise in ordered fashion from left to right implying that each row's
    section within the column-index array is sorted in ascending fashion. Additionally, the row-pointer array contains
    an additional element denoting the total number of non-zero elements in the structure. Note that sometimes a
    different nomenclature is utilized in the existing literature, referring to the arrays as A (values), JA
    (column-indices) and IA (row-pointers), respectively.

    Good properties for mat-vec-mult (locality)

    Compressed Sparse Column

  \subsection{Structured Grid Matrices}
    Arise as finite difference adjacency matrices from discretization of elliptic type PDE (SPARSKIT 7.1)
      and structured mesh finite element matrices.
    Are (locally) structured.
    "Almost" diagonal --> DIA not the best. [Godwin2013: p9!!!]

\section{Three-fold compressed sparse row}
  Improve upon CSR format utilizing domain knowledge about structured grid matrices.

  \subsection{Data layout and storage scheme}
    Starts out at a CSR representation. This section explains the general storage scheme. Its usage and motivation will be clarified in the next section.
    Applies compression idea of CSR to the non-zeros' value array and column index array separately (... motivation?).
    V stores the numerical value of each non-zero entry, VS stores an index-pointer into V to each initial element of a row's section within V.

      Example: Compressed and uncompressed values (Matrix <-> Varrays)

    Column index information is split into three arrays.
    Peg-index: Column index of each row's first non-zero. (JPeg)
    jrel: Column indices relative to their row's peg-index (:= 'index pattern') (JRel)
    jrelstart: Index-pointer into JRel to each initial element of a row's section within JRel.

      Example: Compressed and uncompressed column indices (Matrix <-> Varrays)

    High-level view of the data organization:
    (A) Data representing the matrix's non-zero elements
      A real array V  and vstart;

    (B) Data representing non-zero elements' column indices
      jrel and jrelstart and peg-index

    (C) Row pointers (as in CSR) row-pointers

  \subsection{Compression mechanism}

  \subsection{Algorithms}

    \subsubsection{Matvecmult}
      improves locality (--> Performance)

    \subsubsection{Element access A[i;j]}
      binary lookup. As we're having fun with \textcite{test}, lets do it.

\section{Performance Benchmarks}
  Measure performance in terms of (1) data compression ratio and (2) arithmetic performance (matvecmult)

  \subsection{Generation of structured grid adjacency matrices as test matrices}

  \subsection{Data compression}

  \subsection{Arithmetic performance (matvecmult)}

\section{Summary}
  Advantages: General purpose, good performance, parallel scalability
  Disadv: Static structure (cannot add/remove elements)

\printbibliography
\end{document}
